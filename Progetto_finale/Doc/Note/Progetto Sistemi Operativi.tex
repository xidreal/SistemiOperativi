\documentclass[10pt,a4paper]{article}
\usepackage[utf8]{inputenc}
\usepackage{amsmath}
\usepackage{amsfonts}
\usepackage{amssymb}
\usepackage{graphicx}
\usepackage{hyperref}
\usepackage{fancyvrb}
\title{Progetto Sistemi Operativi}
\author{Davide Bleggi}
\begin{document}
\maketitle
\newpage
\section{Introdzione}
Trasferimento di messaggi tra dispositivi (\ref{device}). I device si muovono all'interno di una scacchiera.
\subsection{Schema funzionamento}
\includegraphics[scale=1.5]{schema}

\section{Elementi}

\subsection{Device} \label{device}
5 Processi figli del processo server (\ref{server}). Ognuno gestisce la propria FIFO (\ref{ClientDevice}).
Invia i messaggi che ha ( e che gli altri non hano) agli altri device nell'area del suo raggio di comunicazione ({\ref{RaggioAzione}) e così faranno anche gli altri.
I dispositivi devono poter memorizzare e gestire più messaggi contemporaneamente.
\subsubsection{Funzionamento}
\begin{enumerate}
\item Controllo dell'acknowledgement (\ref{acknowledgement}) per verificare a quale device il messaggio non sia ancora arrivato.
\item Invio dei messaggi (se disponibili) tramite le relative FIFO dei messaggi.
\item Ricezione di messaggi
\item Movimento
\end{enumerate}
\subsubsection{Device $\longrightarrow$ Device} 
Il nome della FIFO è \emph{dev\_fifo.pid}. La fifo è contenuta in \emph{/tmp/dev\_fifo.pid}. Deve sempre rimanere aperta.
\paragraph{Messaggio} 
La struttura del messaggio inviato è:
\begin{itemize}
\item \emph{pid\_sender}: pid del device
\item \emph{pid\_receiver}: pid del del device ricevente
\item \emph{message\_id}: id del messaggio
\item \emph{message}: stringa di testo.
\item \emph{max\_dist}: numero positivo = raggio di invio del messaggio. \label{RaggioAzione}
\end{itemize}

\subsection{Server} \label{server}
Processo padre dei device \ref{device} e di Ack\_manager \ref{AckManager}.
Genera i segmenti di memoria relativi a acknowledge \ref{acknowledgement} e board \ref{board}.
Crea i semafori per l'accesso ai segmenti di memoria in ackowledge, in board e al cambio posizone (movimento).  
Termina solo con SIGTERM (\ref{sigterm}). 
Scandisce il tempo dei movimenti (fa muovere device1 ogni due secondi che farà partire tutti gli altri a cascata).
Ogni due secondi stampa le posizioni dei devices e gli id dei messaggi in essi contenuti.
\paragraph{Esempio stampa}
\begin{verbatim}
# Step i: device positions ########################
pidD1 i_D1 j_D1 msgs: lista message_id
pidD2 i_D2 j_D2 msgs: lista message_id
pidD3 i_D3 j_D3 msgs: lista message_id
pidD4 i_D4 j_D4 msgs: lista message_id
pidD5 i_D5 j_D5 msgs: lista message_id
#############################################
\end{verbatim}
\paragraph{Avvio}
\begin{verbatim}
./server msg_queue_key file_posizioni
\end{verbatim}
\subsubsection{SIGTERM} \label{sigterm}
\begin{itemize}
\item Termina processi devices (\ref{device})
\item Termina ack\_manager (\ref{AckManager})
\item Termina coda di messaggi (\ref{AckManager})
\item Termina le FIFO (\ref{device})
\item termina memoria condivisa (\ref{board}, \ref{acknowledgement})
\item Termina semafori
\end{itemize}

\subsection{Ack\_manager} \label{AckManager}
Processo figlio del processo Server\ref{server} Gestisce la lista condivisa di ackowlodgement (\ref{acknowledgement}). 
Scandisce ad intervalli regolari di 5 secondi la lista  \ref{acknowledgement} per controllare se tutti i dispositivi hanno ricevuto il messaggio.
In caso positivo invia subito la lista di acknowledgements al Client (\ref{AckManagerClient}).
Ack\_manager comunica con Client tramite coda di messaggi.
"Libera" , contrassegnando, i messaggi coinvolti dalla lista condivisa(\ref{acknowledgement}).
\\Ogni cliente deve ricevere la lista relativa al messaggio che ha immesso nel sistema. 
\subsubsection{Ack\_manager $\rightarrow$ Client} \label{AckManagerClient}
Il nome della coda di messaggi è \emph{msg\_queue}

\subsection{Acknowledgment List} \label{acknowledgement}
Segmento di memoria condivisa generato da server \ref{server}. Gestisce il tracciamento di messaggi tra devices. 
\\Numero finito di messaggi contenibili.
\paragraph{Messaggio} 
La struttura del messaggio inviato è:
\begin{itemize}
\item \emph{pid\_sender}: pid del device
\item \emph{pid\_receiver}: pid del del device ricevente
\item \emph{message\_id}: id del messaggio
\item \emph{date\_time}: data e ora di un passaggio
\end{itemize}
\paragraph{Struttura dati messaggio}
Quindi la struttura dati è:
\begin{Verbatim}
typedef struct {
	pid_t pid_sender;
	pid_t pid_receiver;
	int message_id;
	time_t timestamp;
} Acknowledgment;
\end{Verbatim}

\subsection{Board (Scacchiera)} \label{board}
Scacchiera 10x10. Segmento di memoria condivisa generato da server \ref{server}. In posizione i,j ha scritto il PID del device (\ref{device}) che è in quella posizione. Default cella = 0. I movimenti dei devices sulla scacchiera avvengono a turno ogni 2 secondi (tempo dato dal server). La sincronizzazione dei movimenti avviene tramite semaforo (\ref{semaforo}).
\subsubsection{Posizioni}
File posizione. Direttive di spostamento dei device.
\paragraph{formato}
\begin{Verbatim}
1,7|7,3|2,6|1,2|7,2
0,7|7,4|3,6|1,1|7,1
...
\end{Verbatim}
Ciascuna riga rappresenta la posizione (coordinate x,y, dove x è la riga ed y la colonna) dei 5 device nella scacchiera in un certo istante
\subsubsection{Semaforo} \label{semaforo}
Il semaforo si chiama \emph{SEM\_IDX\_BOARD}.

\subsection{Client}
Processo generato dall'utente. 
Il client comunica con il Device tramite FIFO (\ref{ClientDevice}).
Più client possono inviare messaggi contemporaneamente ai dispositivi.
Quando riceve il messaggio da parte di ack\_manager (\ref{AckManager}), genera un file di nome \emph out\_message\_id.txt(\ref{outputfile}) dove message\_id è l'id del messaggio.
Una volta generato il file il client termina.
\textbf{\\Il message\_id deve essere univoco.}
\paragraph{Avvio}
\begin{verbatim}
./client msg_queue_key
\end{verbatim}
\paragraph{Richiesta client $ \rightarrow$ utente}
\begin{itemize}
\item Inserire pid device a cui inviare il messaggio (pid\_t pid\_device)
\item Inserire id messaggio (int message\_id)
\item Inserire messaggio (char* message)
\item Inserire massima distanza comunicazione per il messaggio (double max\_distance)
\end{itemize}
\subsubsection{Output file}  \label{outputfile}
Lista di 5 acknowledgement che identificano i passaggi fatti dal messaggio con i relativi istanti di tempo.
\paragraph{Formato}
\begin{Verbatim}
Messaggio ‘message_id’: ‘message’
Lista acknowledgment:
pid_client, pid D1, date_time
pid D1, pid D2, date_time
pid D2, pid D3, date_time
pid D3, pid D4, date_time
pid D4, pid D5, date_time
\end{Verbatim}
\subsubsection{Client $\longrightarrow$ Device} \label{ClientDevice}
Il nome della FIFO è \emph{dev\_fifo.pid}.
\paragraph{Messaggio} 
La struttura del messaggio inviato è:
\begin{itemize}
\item \emph{pid\_sender}: pid del client
\item \emph{pid\_receiver}: pid del del device ricevente
\item \emph{message\_id}: id del messaggio
\item \emph{message}: stringa di testo.
\item \emph{max\_idist}: numero positivo = raggio di invio del messaggio. 
\end{itemize}
\paragraph{Struttura dati messaggio}
La struttura dati è quindi:
\begin{Verbatim}
typedef struct {
	pid_t pid_sender;
	pid_t pid_receiver;
	int message_id;
	char message[256];
	int max_distance;
} Message;
\end{Verbatim}

\section{Ipotesi}
\subsection{Client}
\paragraph{Input arg}
\emph{msg\_queue\_key}

\paragraph{Pseudo codice}
\begin{Verbatim}
// Struttura
struct message = {pid_t pid_sender, pid_t pid_receiver, 
int message_id, char message[256], int max_dist) 

// Scrittura del messaggio
Crea this_message;
Chiedi informazioni(this_message);
my_message_id = this_message.message_id;
Dev_FIFO = Apri in scrittura la "dev_fifo".(FIFO this_message.pid_receiver);
Scrivi su Dev_FIFO (this_message);

// Lettura conferma
Crea msg_queue(msg_queue_key);
do{
	Attendi in lettura sulla msg_queue -> acknowledge *Acknowlodge_list;
}while (Acknowlodge_list[0].message_id != this_message.message_id);

// Scrittura su file
file = crea "out"_(this_message.message_id);
while (Acknowlodge_list[n] != NULL)
	Stampa su file(Acknowlodge_list[n]);

Chiudi FIFO;
Chiudi file;
Termina;
\end{Verbatim}

\subsection{Server}
\paragraph{Input arg}
\emph{msg\_queue\_key, file\_posizioni}
\paragraph{Pseudo codice}

\begin{Verbatim}[tabsize = 3]
// Scacchiera
Board = pid_t [10][10];
Key Board_memory = Genera segmento di memoria Board;
Crea semaforo SEM_IDX_BOARD(0,0,0,0,0);
// Acknowledge_list
Crea semaforo SEM_IDX_ACK;
Key Acknowledge_memory = Genera segmento di memoria Acknowlodge_list[100];
pid_t Device[5];

// Apri file posizioni
Apri file_posizioni;

// Crea 5 processi figlio
for (d = 0; d < 5, d++){
	Device[d] = fork();	
	
	// Crea un processo figlio
	if (Device[d] == 0){
		// Crea la fifo 
		Crea la FIFO "/tmp/dev_fifo".getpid();
		// !Inizializzo struct message messaage_list[100];
		// continua finche non viene terminato dal server
		while(1){
			// Crea il proprio posto nella scacchiera
			Leggi da file_posizioni i,j lo spazio d-esimo;			
			Board[i][j] = getpid();
			
			Controlla SEM_IDX_ACK;
					
			Leggo la FIFO aperta;			
			// controllo se nel raggio d'azione 
				sono presenti altri device
			n = 0;
			while (message_list[n+1] != NULL){
				// Controlla la scacchiera alla ricerca delle
				// poszioni degli altri device
				for(i = 0; i < 10; i++)
					for (j = 0; j <10; j++)
						// Controlla il device trovato	
						if(Board[i][j] == message_list[n].pid_receiver && 
							sqrt((i*i + j*j)) < message_list[n].max_dist){
							
							// invia messaggio e scrvilo su acknowledge list
							Controlla SEM_IDX_ACK;						
							Scrivi message_list[n] 
								sulla FIFO "/tmp/dev_fifo".list[n].pid_receiver;
							Libera SEM_IDX_ACK;	
							Crea un new_acknowledgement con message_list[n];					
							Scrivi  new_acknowledgement su Acknowledge_list;
						}
				n++;
							
			}
			Libera SEM_IDX_ACK del device successivo;				
			// DA CREARE HANDLER per la rimozione 
			// della FIFO relativa al processo
		}
}

struct identify {
	int message_id;
	pid_t pid[5] = {0};
}
ack_mager =fork();
if(ack_manager == 0){
	while(1){
		Crea message_queue (message_queue_key);
		Legge acknowledgement_list;
		struct indentify Array_code[100];
		while (acknowledgement_list[i] != NULL)
			while( Array_code[j].mesage_id != acknowledgement_list[i].message_id)
				j++;
				if ( Array_code[j] == NULL){
					Array_code[j].message_id =  acknowledgement_list[i].message_id;
					for(n = 0; n < 5; n++){
						if (Array_code[j].pid == Device[n]){
							Array_code[j].pid[n]++;
						if (Array_code[j].pid[n] > 0){
							counter++;						
						}
						if (couter == 5){
							invia tutti gli acknowledge con questo message_id al client;
							Contrassegna come letti i rimovibili gli acknowledge;						
						}
						}	
				}
			if(Array_code[j].mesage_id == acknowledgement_list[i].message_id){
					for(n = 0; n < 5; n++){
						if (Array_code[j].pid == Device[n]){
							Array_code[j].pid[n]++;
						}	
					}		
			}
			
						
			i++;
		}
 		// DA CREARE HANDLER per la rimozione della message queue;
 	}
	
}

while (ci sono righe in file poszioni){
	print("# Step i: device positions ########################");\\
	
	for(i = 0; i < 10; i++)
		for (j = 0; j < 10; j++)
			if(Board[i][j] == D[n])
				leggi FIFO relativa a D[n];
				print (D[n] i j msgs: lista message_id);

	print("#############################################");
	sleep(2);
	SEM_IDX_BOARD[n](1,0,0,0,0);
	n = 0;
}	

		
		
Termina Ack_manager;
Termina Devices,
Rimuovi SEM_IDX_BOARD;
Rimuovi Acknowledge_access;
Rimuovi Board_memory;
Rimuovi Acknowledge_memory;
Termina;
\end{Verbatim}

\end{document}
