\documentclass[10pt,a4paper]{article}
\usepackage[utf8]{inputenc}
\usepackage{amsmath}
\usepackage{amsfonts}
\usepackage{amssymb}
\usepackage{graphicx}
\usepackage{hyperref}
\title{Progetto Sistemi Operativi}
\begin{document}
\maketitle
\newpage
\section{Introdzione}
Trasferimento di messaggi tra dispositivi (\ref{device}). I device si muovono all'interno di una scacchiera.
\subsection{Schema funzionamento}
\includegraphics[scale=1.5]{schema}

\section{Elementi}
\subsection{Device} \label{device}
5 Processi figli del processo server (\ref{server}). Ognuno gestisce la propria FIFO (\ref{ClientDevice}).
Invia i messaggi che ha agli altri device nell'area del suo raggio di comunicazione ({\ref{RaggioAzione}) e così anche gli altri.
I dispositivi devono poter memorizzare e gestire più messaggi contemporaneamente.
\subsubsection{Funzionamento}
\begin{enumerate}
\item invio dei messaggi (se disponibili)
\item ricezione di messaggi
\item movimento
\end{enumerate}
\subsubsection{Device $\longrightarrow$ Device} 
Il nome della FiFO è \emph{dev\_fifo.pid}.
\paragraph{Messaggio} 
La struttura del messaggio inviato è:
\begin{itemize}
\item \emph{pid\_sender}: pid del device
\item \emph{pid\_receiver}: pid del del device ricevente
\item \emph{message\_id}: id del messaggio
\item \emph{message}: stringa di testo.
\item \emph{max\_dist}: numero positivo = raggio di invio del messaggio. \label{RaggioAzione}
\end{itemize}
\subsubsection{Acknowledgment List} \label{acknowledgement}
Segmento di memoria condivisa generato da server \ref{server}. Gestisce il tracciamento di messaggi tra devices. 
\paragraph{Messaggio} 
La struttura del messaggio inviato è:
\begin{itemize}
\item \emph{pid\_sender}: pid del device
\item \emph{pid\_receiver}: pid del del device ricevente
\item \emph{message\_id}: id del messaggio
\item \emph{date\_time}: data e ora di un passaggio
\end{itemize}

\subsection{Server} \label{server}
Processo padre dei device \ref{device} e di Ack\_manager \ref{AckManager}.
Genera i segmenti di memoria relativi a acknowledge \ref{acknowledgement} e board \ref{board}.
Crea i semafori per l'accesso ai segmenti di memoria in ackowledge, in board e al cambio posizone (movimento).  
Termina solo con SIGTERM (\ref{sigterm}). 
Scandisce il tempo dei movimenti.
\subsubsection{SIGTERM} \label{sigterm}
\begin{itemize}
\item Termina processi devices (\ref{device})
\item Termina ack\_manager (\ref{AckManager})
\item Termina coda di messaggi (\ref{AckManager})
\item Termina FIFO (\ref{device})
\item termina memoria condivisa (\ref{board}, \ref{acknowledgement})
\item Termina semafori
\end{itemize}

\subsection{Ack\_manager} \label{AckManager}
Processo figlio del processo Server\ref{server} Gestisce la lista condivisa di ackowlodgement (\ref{acknowledgement}). 
Scandisce ad intervalli regolari di 5 secondi la lista per controllare se tutti i dispositivi hanno ricevuto il messaggio.
In caso positivo invia subito la lista di acknowledgements al client (\ref{AckManagerClient}).
Ack\_manager comunica con Client tramite coda di messaggi.
Rimuove i messaggi dalla lista condivisa(\ref{acknowledgement}).
\subsubsection{Ack\_manager $\rightarrow$ Client} \label{AckManagerClient}
Il nome della coda di messaggi è \emph{msg\_queue}

\subsection{Board (Scacchiera)} \label{board}
Scacchiera 10x10. Segmento di memoria condivisa generato da server \ref{server}. In posizione i,j ha scritto il PID del device (\ref{device}) che è in quella posizione. Default cella = 0. I movimenti dei devices sulla scacchiera avvengono a turno ogni 2 secondi (tempo dato dal server). La sincronizzazione dei movimenti avviene tramite semaforo (\ref{semaforo}).
\subsubsection{Posizioni}
File posizione. Direttive di spostamento dei device.
\subsubsection{Semaforo} \label{semaforo}
Il semaforo si chiama \emph{SEM\_IDX\_BOARD}.

\subsection{Client}
Processo generato dall'utente. 
Il client comunica con il Device tramite FIFO (\ref{ClientDevice}).
Più client possono inviare messaggi contemporaneamente ai dispositivi.
Quando riceve il messaggio da parte di acknowledgement (\ref{acknowledgement}), genera un file di nome \emph out\_message\_id.txt(\ref{outputfile}) dove message\_id è l'id del messaggio.
Una volta generato il file il client termina.
\subsubsection{Output file}  \label{outputfile}
Lista di 5 acknowledgement che identificano i passaggi fatti dal messaggio con i relativi istanti di tempo
\subsubsection{Client $\longrightarrow$ Device} \label{ClientDevice}
Il nome della FIFO è \emph{dev\_fifo.pid}.
\paragraph{Messaggio} 
La struttura del messaggio inviato è:
\begin{itemize}
\item \emph{pid\_sender}: pid del client
\item \emph{pid\_receiver}: pid del del device ricevente
\item \emph{message\_id}: id del messaggio
\item \emph{message}: stringa di testo.
\item \emph{max\_idist}: numero positivo = raggio di invio del messaggio. 
\end{itemize}


\end{document}